% !TEX root =  manual.tex
\section{Emulator Tuning}\label{sec:emulatortuning}

As part of training the emulator, \path{madai\_train\_emulator} determines the hyper-parameters ($\Theta_k$ in Eq. \eqref{eq:hyperform}) to be used by the emulator. In many cases, these parameters are simply chosen \emph{ad hoc}, but \path{madai\_train\_emulator} initializes them to reasonable values. To train the emulator, run the command

\commandline{madai\_train\_emulator stat1}

When training the emulator, \path{madai\_train\_emulator} looks for parameters defined in a file named\\ \path{stat1/settings.dat}, which is a text file describing parameters used in the emulator construction and tuning and for the MCMC procedure. The file has a simple form, 

\begin{quote}
{\tt MCMC\_USE\_MODEL\_ERROR             false}\\
{\tt PCA\_FRACTION\_RESOLVING\_POWER     0.99}\\
{\tt MCMC\_NUMBER\_OF\_BURN\_IN\_SAMPLES 100000}\\
{\tt SAMPLER\_NUMBER\_OF\_SAMPLES           1000000}\\
{\tt MCMC\_STEP\_SIZE                    0.1}
\end{quote}

\path{madai\_train\_emulator} produces a file called \path{stat1/EmulatorState.dat}. This file has the following format:
\begin{quote}
{\tt SUBMODELS}~$M_Z$~~~$\leftarrow$Each principal component is modeled separately.\\
{\tt MODEL~~~0}~~~~~~~~~$\leftarrow$ The submodels are numbered zero through $M_z-1$.\\
{\tt COVARIANCE\_FUNCTION} {\it SQUARE\_EXPONENTIAL\_FUNCTION}\\
$\uparrow$ This describes the  functional form assumed by the GPE for the covariance.\\
The choice determines what hyper-parameters mean.\\
{\tt REGRESSION\_ORDER }~~ $\leftarrow$ Order of polynomial to fit to the model outputs\\
{\tt THETAS}~~~~~$\leftarrow$ Hyper-parameter values for this submodel\\
$N_\Theta$~~~~~~~~~~$\leftarrow$ This is the number of hyper-parameters\\
$\Theta_1$\\
$\Theta_2$\\
$\vdots$\\
$\Theta_{N_\Theta}$\\
{\tt END\_OF\_MODEL}\\
{\tt MODEL~~1}\\
$\vdots$\\
{\tt END\_OF\_MODEL}\\
{\tt MODEL}~~~{$M_z-1$}\\
$\vdots$\\
{\tt END\_OF\_MODEL}\\
{\tt END\_OF\_FILE}
\end{quote}\vspace*{-8pt}

The $\Theta_i$ values can be modified in these files to tune the emulator.

There are several supported choices for the \variable{COVARIANCE\_FUNCTION}. See section \ref{sec:CovarianceFunctions}.
