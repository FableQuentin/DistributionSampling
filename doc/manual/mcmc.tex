% !TEX root =  manual.tex
\section{Markov Chain Monte Carlo}

Scott does this part

After the emulator has been tuned, you can start to generate traces from it. The command to generate a trace is

\commandline{generateMCMCtrace stat1 trace.csv}

This command reads in the files stored in \path{stat1}, runs a Markov Chain Monte Carlo (MCMC) algorithm to explore the high-dimensional model parameter space, and produces a file named \path{trace.csv} in comma-separated value format (CSV). The \path{trace.csv} file is saved to the directory \\\path{stat1/MCMCTrace/trace.csv}.

Each line of the CSV file contains information from one sample generated by the MCMC algorithm. The first entries in the line are the parameter in the model parameter space. The next entries in the line are the model outputs produced by running the model at that point in parameter space. The last entry is the log-likelihood that indicates how well the model outputs match the experimental data.

A brief sample from a CSV file created by \path{generateMCMCtrace}  is shown below:

\begin{verbatim}
"p1","p2","p3","o1","o2","o3","o4","LogLikelihood"
1.19431,3.81693,0.604379,0.752093,3.88979,1.09151,0.907032,-1.44567
1.18372,3.8114,0.600692,0.747385,3.88466,1.08613,0.902906,-1.4521
1.17554,3.78698,0.610058,0.74376,3.86122,1.09977,0.899749,-1.46829
1.16455,3.77324,0.600326,0.738865,3.8481,1.08558,0.895453,-1.48468
1.17112,3.77779,0.592067,0.741777,3.85229,1.07355,0.897992,-1.51059
1.15852,3.78186,0.582758,0.736163,3.85649,1.05998,0.893065,-1.589
\end{verbatim}
