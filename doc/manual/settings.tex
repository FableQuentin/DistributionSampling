% !TEX root =  manual.tex

\section{Settings File}\label{sec:settings}

The file \path{settings.dat} controls the operation of most of the
Distribution Sampling Programs.

\begin{description}
    \item[MODEL\_OUTPUT\_DIRECTORY] (default: model\_output) The directory (relative to the working directory) where the model inputs and outputs can be found.
    \item[EXPERIMENTAL\_RESULTS\_FILE] (default: experimental\_results.dat)
    \item[VERBOSE] (default: 0) Print warnings and other potentially useful information. When this option is set to false, the programs will exit silently when everything has succeeded.
    \item[READER\_VERBOSE] (default: 0) Print copious amounts of information while reading the data directory structure.
    \item[GENERATE\_TRAINING\_POINTS\_NUMBER\_OF\_POINTS] (default: 100) The number of training points generated by madai\_generate\_training\_points
    \item[GENERATE\_TRAINING\_POINTS\_PARTITION\_BY\_PERCENTILE] (default: 1)
    \item[GENERATE\_TRAINING\_POINTS\_STANDARD\_DEVIATIONS] (default: 3) If the prior distribution for a parameter is Gaussian, how far out should the uniformly-spaced sampled points go?
    \item[GENERATE\_TRAINING\_POINTS\_USE\_MAXIMIN] (default: false) Controls whether madai\_generate\_training\_points uses the maximin algorithm to generate the training points. This algorithm attempts to find a Latin hypercube sampling that has a large minimimum distance between points in parameter space.
    \item[GENERATE\_TRAINING\_POINTS\_MAXIMIN\_TRIES] (default: 20) When GENERATE\_TRAINING\_POINTS\_USE\_MAXIMIN is on, this option determines the number of Latin hypercubes that are generated when searching for one with the largest minimum distance between points in parameter space.
    \item[GENERATE\_POSTERIOR\_POINTS\_NUMBER\_OF\_POINTS] (default 20) After performing the MCMC trace, one may run the program \path{madai\_generate\_posterior\_points} to generate a sample of points consistent with the posterior distribution.
    \item[PCA\_FRACTION\_RESOLVING\_POWER] (default: 0.95)  When madai\_train\_emulator is deciding how many principal components to retain, it considers the total resolving power of the model. The resolving power of a single principal component is $\sqrt{(1 + \lambda_i)}$, where $\lambda_i$ is the eigenvalue associated with that variable. The total resolving power is the product the resolving powers of all of the components. madai\_train\_emulator will retain enough components to keep this fraction of the resolving power.
    \item[EMULATOR\_COVARIANCE\_FUNCTION] (default: SQUARE\_EXPONENTIAL\_FUNCTION) See section \ref{sec:CovarianceFunctions}.
    \item[EMULATOR\_REGRESSION\_ORDER] (default: 1) The assumed functional form of the model before Gaussian Process Emulation. This is the order of the polynomial. 0, 1, 2,or 3.
    \item[EMULATOR\_NUGGET] (default: 0.001) $\theta_1$. See section \ref{sec:CovarianceFunctions}.
    \item[EMULATOR\_AMPLITUDE] (default: 1) $\theta_0$. See section \ref{sec:CovarianceFunctions}.
    \item[EMULATOR\_SCALE] (default: 0.01) See section \ref{sec:CovarianceFunctions}. The final value for each parameter's scale ($\theta_{2+i}$) will be EMULATOR\_SCALE times the width of the middle two quartiles of that parameter's prior distribution.
    \item[EMULATOR\_TRAINING\_RIGOR] (default: basic) Determines the rigor with which the emulator is trained. Currently, only the default value of ``basic'' is supported.
    \item[SAMPLER] (default: ``MetropolisHastings'') Determines which sampler to use to generate sample points. Available samplers are ``MetropolisHastings'' and ``PercentileGrid''.
    \item[SAMPLER\_NUMBER\_OF\_SAMPLES] (default: 100) How many samples should be generated in a trace. The default is small for testing purposes. Values around $10^6$ are recommended.
    \item[MCMC\_USE\_MODEL\_ERROR] (default: 0) Specifies whether error reported by the model should be used in the log likelihood calculation. Turning this off may make computation of the log likelihood faster at the cost of assuming that the model error is zero for each output.
    \item[MCMC\_NUMBER\_OF\_BURN\_IN\_SAMPLES] (default: 0) The number of MCMC samples to be discarded at the beginning of the MCMC run.
    \item[MCMC\_STEP\_SIZE] (default: 0.1) How big should each step be in the Metropolis Hastings algorithm? (This will be scaled by the characteristic length of each parameter's prior distribution)
    \item[EXTERNAL\_MODEL\_EXECUTABLE] (default: none) Path to an external executable that follows the protocol for writing to stdout and reading from stdin assumed by the \path{madai\_generate\_trace} program. If this option is not set, \path{madai\_generate\_trace} uses a trained emulator to generate the trace.
    \item[EXTERNAL\_MODEL\_ARGUMENTS] (default: none) Arguments to pass to the executable pointed to by EXTERNAL\_MODEL\_EXECUTABLE.
    \item[EMULATE\_QUIET] (default: 0)
    \item[EMULATE\_WRITE\_HEADER] (default: 1)
    \item[POSTERIOR_ANALYSIS_DIRECTORY] (default: posterior_model_output) Specifies the directory where madai_generate_posterior_samples generates its results. If specified as a relative path, the path is considered to be relative to the statistica analysis directory.
\end{description}

